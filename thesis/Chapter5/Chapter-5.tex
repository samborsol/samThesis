
\chapter{Trigger Development and Performance}

\section{Introduction to Triggering}

Modern "triggering" methods began with Walther Bothe's development of the coincidence circuit. The coincidence ciruit accepts two inputs. If these inputs are recieced within the same time window, approximating coincidence, the circuit passes an output. Bothe's originally used the coincidence circuit to take data for electron-photon production in Compton scattering. Figure \ref{fig:ross} is the diagram of a coincidence circuit. 

\begin{figure}[h!]
\begin{centering}
\includegraphics[width=4in]{Chapter5/importfigs/Rossis-coincidence-circuit-appearing-in-Ref-65-The-selecting-resistance-on-the-right.png}
\par\end{centering}
\caption{Rossi circuit with three Geiger-Muller coincidence circuits. \label{fig:ross}}
\end{figure}

\section{Triggering at CMS}

At stable beams, the LHC delivers bunch crossings every 25 nanoseconds. Each bunch crossing in turn will have some 20 hadron collisions. The resulting interaction rate -- $10^9$ interactions per second -- is orders of magnitude greater than the frequency that events can be written to disk, $10^2$ events per second. CMS therefore needs a means of filtering out the most interesting $10^2$ interactions per second while declining the other $10^6$ interactions per second. 

CMS uses a two-tiered triggering system. The first tier, the L1 trigger, is hardware based. The second tier, the high-level trigger (HLT), is software based. The L1 trigger recieves raw data from the calorimeters and the muon detectors; this determines when the tracker will readout data. The raw data from the tracker, calorimeters, and muon detectors is then passed on to a computer farm running the HLT menu. The HLT then performs a simplistic reconstruction of the raw data into physics objects useful for analysis: jets, tracks, and identifiable particles. If an event passes the HLT, the raw data is permanently stored in preparation for a more complex reconstruction. 

The initial interaction rate is approximately $3.2$ $\mu s$. The L1 trigger can only pass some 1 in 1000 interactions to the HLT. The L1 trigger menu has an output rate of approximately 100 kHz. L1 achieves this rate by only considering data of reduced granularity and reduced resolution. A buffer is used to store the full event data while the L1 runs. The HLT menu reduces this to about 100 Hz as required by the limit on disk writing. 

The 2015 UPC triggers were for low multiplicity events and low transverse momentum events. Typical heavy-ion collisions are high multiplicity events. Fig.\ref{fig:eventdisplayHI} is an event display of one of the first heavy-ion collisions at CMS in 2010.

\begin{figure}[h!]
\begin{centering}
\includegraphics[width=4in]{Chapter3/importfigs/cms_firstleadcoll.jpg}
\par\end{centering}
\caption{High multiplicity PbPb collision \label{fig:eventdisplayHI}}
\end{figure}

Fig.\ref{fig:eventdisplayUPCUps} is the event display of a UPC upsilon candidate. Notice that there are only two reconstructed tracks, and that CMS is otherwise empty in all calorimeters. 

\begin{figure}[h!]
\begin{centering}
\includegraphics[width=4in]{Chapter3/importfigs/upcJpsi_run285530_lumi594_event944509077_v0.png}
\par\end{centering}
\caption{UPC Upsilon candidate \label{fig:eventdisplayUPCUps}}
\end{figure}

For this analysis, the L1 trigger applies two selections. First, the L1 checks that at least one of the HF is empty. This is the most important part of the trigger in so far as it suppressed the hadronic contamination of the dataset. Then, if there is at least 5 GeV of energy deposited in the ECAL, the event passes to the HLT. 

Low multiplicity events are difficult to distinguish from background. To compensate, the HLT in turn requires that there be at least once reconstructed track from the pixel tracker, to make sure that there are particles that will be reconstructed by the complete tracker. Only the pixel tracker is used for these HLTs to increase the speed of reconstruction while decreasing needed computer cycles. 

\section{Author's Contributions}

In preparation for the 2015 heavy-ion run and the 2016 p-Pb run, I prepared high-level trigger menus for the CMS Forward-HI group. This trigger menu was optimized for firing on ultra-peripheral collisions. I tested the menu's performance on Monte Carlo generated by STARLIGHT and reconstructed through a GEANT4 simulation of CMS. During the experiment, I was present at CERN to monitor the trigger rates and deliver daily reports on their performance. 

CMSSW includes an emulator for the L1 trigger. This software can re-emulate alternative L1 menus on previously taken CMS data. The performance of a trigger on the 2011 PbPb data, taken at $\sqrt{s_NN} = 2.7 TeV$ is extrapolated to the higher energy, $\sqrt{s_NN} = 5.2 TeV$, of the 2015 PbPb run.

I tested my HLT menu on both STARLIGHT MC and on data from the 2011 Pb+Pb run. STARLIGHT is a MC generator for ultra-peripheral collisions, in particular for the vector-meson photoproduction channels. I used STARLIGHT to generate MC sets for $Pb+Pb\rightarrow J/\Psi+Pb+Pb$ and $Pb+Pb\rightarrow \Upsilon(1s)+Pb+Pb$. I then used CMSSW to test the performance of the component bits of the HLT paths with respect to these MC sets. 

\begin{figure}[h!]
\begin{centering}
\includegraphics[width=4in]{Chapter5/importfigs/triggerRateExample.png}
\par\end{centering}
\caption{Example of trigger rate. \label{fig:trigRate}}
\end{figure}

It was my duty to carefully observe the state of the UPC triggers during the heavy-ion run. If the total HLT rate ever drifted above 100 Hz, the HLT menu could crash and CMS lose considerable data. It was important for the trigger contacts to make sure that there HLT paths were behaving stablely. I also analyzed express physics data to test that the vector-meson triggers saw appropriate mass resonances.


\section{Studies from Trigger Menus}

The UPC HLTs I designed are useful for a variety of physics studies. In particular, vector meson photoproduction, light-by-light scattering, and UPC particle correlations. 

The single and double muon UPC triggers are adequate for studies of vector-meson photoproduction. As of this writing, the Forward-HIN group has ongoing studies of UPC $J/\Psi$ and UPC $\Upsilon(1s)$ in both the 2015 PbPb and 2016 pPb data. 