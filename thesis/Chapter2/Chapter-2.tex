
\chapter{Diffractive Dijet Photoproduction}

Diffractive dijet photoproduction is not describable in perturbative QCD. For coherent processes the photon energy is small, and therefore the wavelength is large compared to the size of the nucleus. At these large distances, there isn't a hard scale, and so perturbation calculations cannot be done. Gluon splitting interactions dominate the low Bjorken-x partons. QCD collinear factorization describes these soft interactions via the convolution of parton cross sections, taken from perturbative QCD, and diffractive parton distribution functions, taken from experiment. 

In electron-hadron collisions, diffractive photoproduction is characterized by the presence of a large rapidity gap in the final state and an intact nucleus. The Feynman diagram of electroproduction in lepton-hadron collisions is similar to that of photoproduction in ultraperipheral collisions.

Lepton-hadron collisions were performed at DESY and measured by the H1 and HERA experiments. These experiments reported a value for the total diffractive photoproduction cross section that is double that predicted by QCD collinear factorization. Diffractive events were selected for using rapidity gaps or the presence of intact protons in the very forward proton spectrometer (VFPS). 



