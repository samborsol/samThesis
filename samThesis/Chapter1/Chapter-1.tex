
\chapter{Studying the Nuclear PDFs in HI Collisions}

\section{The Standard Model}

The Standard Model describes the fundamental particles of the universe in terms of fermions and bosons. Fermions are particles with half-integer spin, while bosons have integer-spin. This difference in spin has far reaching consequences. Fermions must obey the Pauli Exclusion Principle: only one fermion at a time can occupy a given state. However, multiple bosons can simultaneously occupy a specific state. 

\subsection{Quantum Electrodynamics}

Quantum electrodynamics (QED) is a theory of electromagnetic interaction in terms of relativistic quantum field theory. QED addresses three specific processes: photon motion, electron motion, and the emission, or absorption, of a photon by an electron.

The QED coupling decreases with distance, as manifest the Coulomb force being proportional to an inverse-square law. 

\subsection{Quantum Chromodynamics}

The quarks are a family of fermions that compose the baryons and the mesons. Baryons consist of three quarks in a color neutral state, while mesons consist two quarks in a color neutral state. "Color" in this context refers to the six kinds of strongly-interacting charge available to quarks: red and anti-red, blue and anti-blue, and green and anti-green. Color charge has no relation to optical phenomena, but provides a useful analogy for the stable combinations of quarks. The net color-charge of a baryon or meson is "white".

Unlike QED, the QCD coupling increases with distance. This has the practical consequence of the strong-interactions being stronger in high momentum transfer collisions. The direct results of the running QCD coupling are the dual phenomena of asymptotic freedom and color confinement.

\subsubsection{Asymptotic Freedom}

Within the nucleus, a proton can be thought of as a bubble in a vacuum. Debrye screening exerts a pressure on the proton. This pressure is responsible for the size of the proton. 

\subsubsection{Color Confinement} 

At large distances, string tension describes the binding force of the quarks. At short distances, however, Coulomb-like interactions dominate.

\section{QCD Experiments}

Scattering experiments are the basic tool for exploring the nucleus. The Large Hadron Collider (LHC) is capable of reaching heavy-ion collision energies of up to 7 TeV per nucleon-nucleon. The higher the energy, the more experiments can probe the nuclear phase-space diagram.

At the turn of the century, Ernst Rutherford probed the gold atom by bombarding a gold sheet with alpha-particles (helium nuclei). The angular distribution of the scattered alpha-particles demonstrates that the mass of the atom is concentrated in a small volume, i.e, the atom is mostly empty space. 

Momentum transferred, expressed as $Q^2$, is an important quantity for characterizing QCD measurements. 

In addition to $Q^2$, Bjorken-x, also known as Bjorken-scaling is necessary to describe the nuclear phase space. Bjorken-x represents the momentum fraction of partons. 

\subsection{Hard Processes}

Hard processes involve scattering particles off partons in the manner of point-like charges.

\subsubsection{Deep Inelastic Scattering}

Deep inelastic scattering commonly refers to the scattering of a leptons off hadrons. These experiments provided the first evidence of Bjorken-scaling in the nucleus, a direct interpretation of which is the existence of quarks as point-like particles at high energies.

\subsection{Soft Processes}

Soft processes compose the low momentum transfer, typically gluon-gluon interactions during a  collision.

\subsection{Heavy-Ion Collisions}

Similar to the Rutherford experiment, in heavy-ion collisions the scattered particles carry information about the internal structure of the nucleus. 

The Rutherford experiment has the three components that still characterize high-energy nuclear experiments: a probe, a medium, and a signal. Alpha particles probe the medium of the gold atom, and the angular distribution of scattered alpha particles signals the internal structure of the atom. 

\subsubsection{Ultra-peripheral Collisions}

Ultra-peripheral collisions occur at impact parameters greater than the sum of the heavy-ion radii. In these collisions, hadronic interactions are strongly suppressed while photonuclear activity is enhanced proportional to the square of the nuclear charge. The electromagnetic field of an incoming heavy-ion, from the perspective of a target, is equivalent to a flux of virtual photons. 

\section{Jet Production}

Gluons are the particle exchanged in strong interactions. However, gluons themselves carry color charge. By analogy, photons transmit the electromagnetic force, but do not themselves have an electric charge. 

When a quark is scattered from a nucleus, the strong interaction gathers potential energy until the threshold for quark production is passed, at which point an anti-quark is generated to screen the ejected quark.

\subsection{Diffraction}

QCD factorisation describes the diffractive-photoproduction dijet cross-section as the convolution of the partonic cross-section with the diffractive parton distributions. However, factorisation only describes H1 data if the resolved-photon contribution is suppressed. 

\subsection{Photoproduction}

The photoproduction cross-section is proportional to the gluon distribution.

\centerline{
\includegraphics[width=2in]{Chapter1/importfigs/fig1a.png}
\includegraphics[width=2in]{Chapter1/importfigs/fig1b.png}
}

\subsubsection{Direct Photon Processes}

At low momentum transfer, photons interact electromagnetically, i.e. directly, with partons. 

\subsubsection{Resolved Photon Processes}

High energy photons possess a hadronic structure. 